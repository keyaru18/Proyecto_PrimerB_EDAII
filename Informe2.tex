% Options for packages loaded elsewhere
% Options for packages loaded elsewhere
\PassOptionsToPackage{unicode}{hyperref}
\PassOptionsToPackage{hyphens}{url}
\PassOptionsToPackage{dvipsnames,svgnames,x11names}{xcolor}
%
\documentclass[
  spanish,
  letterpaper,
  DIV=11,
  numbers=noendperiod]{scrartcl}
\usepackage{xcolor}
\usepackage{amsmath,amssymb}
\setcounter{secnumdepth}{-\maxdimen} % remove section numbering
\usepackage{iftex}
\ifPDFTeX
  \usepackage[T1]{fontenc}
  \usepackage[utf8]{inputenc}
  \usepackage{textcomp} % provide euro and other symbols
\else % if luatex or xetex
  \usepackage{unicode-math} % this also loads fontspec
  \defaultfontfeatures{Scale=MatchLowercase}
  \defaultfontfeatures[\rmfamily]{Ligatures=TeX,Scale=1}
\fi
\usepackage{lmodern}
\ifPDFTeX\else
  % xetex/luatex font selection
\fi
% Use upquote if available, for straight quotes in verbatim environments
\IfFileExists{upquote.sty}{\usepackage{upquote}}{}
\IfFileExists{microtype.sty}{% use microtype if available
  \usepackage[]{microtype}
  \UseMicrotypeSet[protrusion]{basicmath} % disable protrusion for tt fonts
}{}
\makeatletter
\@ifundefined{KOMAClassName}{% if non-KOMA class
  \IfFileExists{parskip.sty}{%
    \usepackage{parskip}
  }{% else
    \setlength{\parindent}{0pt}
    \setlength{\parskip}{6pt plus 2pt minus 1pt}}
}{% if KOMA class
  \KOMAoptions{parskip=half}}
\makeatother
% Make \paragraph and \subparagraph free-standing
\makeatletter
\ifx\paragraph\undefined\else
  \let\oldparagraph\paragraph
  \renewcommand{\paragraph}{
    \@ifstar
      \xxxParagraphStar
      \xxxParagraphNoStar
  }
  \newcommand{\xxxParagraphStar}[1]{\oldparagraph*{#1}\mbox{}}
  \newcommand{\xxxParagraphNoStar}[1]{\oldparagraph{#1}\mbox{}}
\fi
\ifx\subparagraph\undefined\else
  \let\oldsubparagraph\subparagraph
  \renewcommand{\subparagraph}{
    \@ifstar
      \xxxSubParagraphStar
      \xxxSubParagraphNoStar
  }
  \newcommand{\xxxSubParagraphStar}[1]{\oldsubparagraph*{#1}\mbox{}}
  \newcommand{\xxxSubParagraphNoStar}[1]{\oldsubparagraph{#1}\mbox{}}
\fi
\makeatother

\usepackage{color}
\usepackage{fancyvrb}
\newcommand{\VerbBar}{|}
\newcommand{\VERB}{\Verb[commandchars=\\\{\}]}
\DefineVerbatimEnvironment{Highlighting}{Verbatim}{commandchars=\\\{\}}
% Add ',fontsize=\small' for more characters per line
\usepackage{framed}
\definecolor{shadecolor}{RGB}{241,243,245}
\newenvironment{Shaded}{\begin{snugshade}}{\end{snugshade}}
\newcommand{\AlertTok}[1]{\textcolor[rgb]{0.68,0.00,0.00}{#1}}
\newcommand{\AnnotationTok}[1]{\textcolor[rgb]{0.37,0.37,0.37}{#1}}
\newcommand{\AttributeTok}[1]{\textcolor[rgb]{0.40,0.45,0.13}{#1}}
\newcommand{\BaseNTok}[1]{\textcolor[rgb]{0.68,0.00,0.00}{#1}}
\newcommand{\BuiltInTok}[1]{\textcolor[rgb]{0.00,0.23,0.31}{#1}}
\newcommand{\CharTok}[1]{\textcolor[rgb]{0.13,0.47,0.30}{#1}}
\newcommand{\CommentTok}[1]{\textcolor[rgb]{0.37,0.37,0.37}{#1}}
\newcommand{\CommentVarTok}[1]{\textcolor[rgb]{0.37,0.37,0.37}{\textit{#1}}}
\newcommand{\ConstantTok}[1]{\textcolor[rgb]{0.56,0.35,0.01}{#1}}
\newcommand{\ControlFlowTok}[1]{\textcolor[rgb]{0.00,0.23,0.31}{\textbf{#1}}}
\newcommand{\DataTypeTok}[1]{\textcolor[rgb]{0.68,0.00,0.00}{#1}}
\newcommand{\DecValTok}[1]{\textcolor[rgb]{0.68,0.00,0.00}{#1}}
\newcommand{\DocumentationTok}[1]{\textcolor[rgb]{0.37,0.37,0.37}{\textit{#1}}}
\newcommand{\ErrorTok}[1]{\textcolor[rgb]{0.68,0.00,0.00}{#1}}
\newcommand{\ExtensionTok}[1]{\textcolor[rgb]{0.00,0.23,0.31}{#1}}
\newcommand{\FloatTok}[1]{\textcolor[rgb]{0.68,0.00,0.00}{#1}}
\newcommand{\FunctionTok}[1]{\textcolor[rgb]{0.28,0.35,0.67}{#1}}
\newcommand{\ImportTok}[1]{\textcolor[rgb]{0.00,0.46,0.62}{#1}}
\newcommand{\InformationTok}[1]{\textcolor[rgb]{0.37,0.37,0.37}{#1}}
\newcommand{\KeywordTok}[1]{\textcolor[rgb]{0.00,0.23,0.31}{\textbf{#1}}}
\newcommand{\NormalTok}[1]{\textcolor[rgb]{0.00,0.23,0.31}{#1}}
\newcommand{\OperatorTok}[1]{\textcolor[rgb]{0.37,0.37,0.37}{#1}}
\newcommand{\OtherTok}[1]{\textcolor[rgb]{0.00,0.23,0.31}{#1}}
\newcommand{\PreprocessorTok}[1]{\textcolor[rgb]{0.68,0.00,0.00}{#1}}
\newcommand{\RegionMarkerTok}[1]{\textcolor[rgb]{0.00,0.23,0.31}{#1}}
\newcommand{\SpecialCharTok}[1]{\textcolor[rgb]{0.37,0.37,0.37}{#1}}
\newcommand{\SpecialStringTok}[1]{\textcolor[rgb]{0.13,0.47,0.30}{#1}}
\newcommand{\StringTok}[1]{\textcolor[rgb]{0.13,0.47,0.30}{#1}}
\newcommand{\VariableTok}[1]{\textcolor[rgb]{0.07,0.07,0.07}{#1}}
\newcommand{\VerbatimStringTok}[1]{\textcolor[rgb]{0.13,0.47,0.30}{#1}}
\newcommand{\WarningTok}[1]{\textcolor[rgb]{0.37,0.37,0.37}{\textit{#1}}}

\usepackage{longtable,booktabs,array}
\usepackage{calc} % for calculating minipage widths
% Correct order of tables after \paragraph or \subparagraph
\usepackage{etoolbox}
\makeatletter
\patchcmd\longtable{\par}{\if@noskipsec\mbox{}\fi\par}{}{}
\makeatother
% Allow footnotes in longtable head/foot
\IfFileExists{footnotehyper.sty}{\usepackage{footnotehyper}}{\usepackage{footnote}}
\makesavenoteenv{longtable}
\usepackage{graphicx}
\makeatletter
\newsavebox\pandoc@box
\newcommand*\pandocbounded[1]{% scales image to fit in text height/width
  \sbox\pandoc@box{#1}%
  \Gscale@div\@tempa{\textheight}{\dimexpr\ht\pandoc@box+\dp\pandoc@box\relax}%
  \Gscale@div\@tempb{\linewidth}{\wd\pandoc@box}%
  \ifdim\@tempb\p@<\@tempa\p@\let\@tempa\@tempb\fi% select the smaller of both
  \ifdim\@tempa\p@<\p@\scalebox{\@tempa}{\usebox\pandoc@box}%
  \else\usebox{\pandoc@box}%
  \fi%
}
% Set default figure placement to htbp
\def\fps@figure{htbp}
\makeatother


% definitions for citeproc citations
\NewDocumentCommand\citeproctext{}{}
\NewDocumentCommand\citeproc{mm}{%
  \begingroup\def\citeproctext{#2}\cite{#1}\endgroup}
\makeatletter
 % allow citations to break across lines
 \let\@cite@ofmt\@firstofone
 % avoid brackets around text for \cite:
 \def\@biblabel#1{}
 \def\@cite#1#2{{#1\if@tempswa , #2\fi}}
\makeatother
\newlength{\cslhangindent}
\setlength{\cslhangindent}{1.5em}
\newlength{\csllabelwidth}
\setlength{\csllabelwidth}{3em}
\newenvironment{CSLReferences}[2] % #1 hanging-indent, #2 entry-spacing
 {\begin{list}{}{%
  \setlength{\itemindent}{0pt}
  \setlength{\leftmargin}{0pt}
  \setlength{\parsep}{0pt}
  % turn on hanging indent if param 1 is 1
  \ifodd #1
   \setlength{\leftmargin}{\cslhangindent}
   \setlength{\itemindent}{-1\cslhangindent}
  \fi
  % set entry spacing
  \setlength{\itemsep}{#2\baselineskip}}}
 {\end{list}}
\usepackage{calc}
\newcommand{\CSLBlock}[1]{\hfill\break\parbox[t]{\linewidth}{\strut\ignorespaces#1\strut}}
\newcommand{\CSLLeftMargin}[1]{\parbox[t]{\csllabelwidth}{\strut#1\strut}}
\newcommand{\CSLRightInline}[1]{\parbox[t]{\linewidth - \csllabelwidth}{\strut#1\strut}}
\newcommand{\CSLIndent}[1]{\hspace{\cslhangindent}#1}

\ifLuaTeX
\usepackage[bidi=basic]{babel}
\else
\usepackage[bidi=default]{babel}
\fi
% get rid of language-specific shorthands (see #6817):
\let\LanguageShortHands\languageshorthands
\def\languageshorthands#1{}


\setlength{\emergencystretch}{3em} % prevent overfull lines

\providecommand{\tightlist}{%
  \setlength{\itemsep}{0pt}\setlength{\parskip}{0pt}}



 


\KOMAoption{captions}{tableheading}
\makeatletter
\@ifpackageloaded{caption}{}{\usepackage{caption}}
\AtBeginDocument{%
\ifdefined\contentsname
  \renewcommand*\contentsname{Tabla de contenidos}
\else
  \newcommand\contentsname{Tabla de contenidos}
\fi
\ifdefined\listfigurename
  \renewcommand*\listfigurename{Listado de Figuras}
\else
  \newcommand\listfigurename{Listado de Figuras}
\fi
\ifdefined\listtablename
  \renewcommand*\listtablename{Listado de Tablas}
\else
  \newcommand\listtablename{Listado de Tablas}
\fi
\ifdefined\figurename
  \renewcommand*\figurename{Figura}
\else
  \newcommand\figurename{Figura}
\fi
\ifdefined\tablename
  \renewcommand*\tablename{Tabla}
\else
  \newcommand\tablename{Tabla}
\fi
}
\@ifpackageloaded{float}{}{\usepackage{float}}
\floatstyle{ruled}
\@ifundefined{c@chapter}{\newfloat{codelisting}{h}{lop}}{\newfloat{codelisting}{h}{lop}[chapter]}
\floatname{codelisting}{Listado}
\newcommand*\listoflistings{\listof{codelisting}{Listado de Listados}}
\makeatother
\makeatletter
\makeatother
\makeatletter
\@ifpackageloaded{caption}{}{\usepackage{caption}}
\@ifpackageloaded{subcaption}{}{\usepackage{subcaption}}
\makeatother
\usepackage{bookmark}
\IfFileExists{xurl.sty}{\usepackage{xurl}}{} % add URL line breaks if available
\urlstyle{same}
\hypersetup{
  pdftitle={Informe: Implementación de BFS y DFS para Análisis de Redes Sociales},
  pdfauthor={Richard Tipantiza, Jairo Angulo , Tamara Benavides},
  pdflang={es},
  colorlinks=true,
  linkcolor={blue},
  filecolor={Maroon},
  citecolor={Blue},
  urlcolor={Blue},
  pdfcreator={LaTeX via pandoc}}


\title{Informe: Implementación de BFS y DFS para Análisis de Redes
Sociales}
\author{Richard Tipantiza, Jairo Angulo , Tamara Benavides}
\date{}
\begin{document}
\maketitle

\renewcommand*\contentsname{Tabla de Contenidos}
{
\hypersetup{linkcolor=}
\setcounter{tocdepth}{3}
\tableofcontents
}

\section{OBJETIVOS}\label{objetivos}

\begin{itemize}
\item
  Desarrollar un grafo de perfiles dentro de una red social,
  implementando estructuras de datos para representar usuarios y
  conexiones, junto con algoritmos de búsqueda BFS y DFS.
\item
  Implementar una estructura de grafo mediante POO, creando clases para
  nodos, aristas y el grafo completo, permitiendo el modelado preciso de
  conexiones en redes sociales.
\end{itemize}

\section{INTRODUCCIÓN}\label{introducciuxf3n}

El proyecto desarrollado implementa los algoritmos BFS (Breadth-First
Search) y DFS (Depth-First Search) para el análisis de conexiones en
redes sociales, donde los nodos representan a los usuarios y las aristas
simbolizan relaciones como amistades o seguidores. Como señala Coto
(2003)(Netto, 2003), estos algoritmos son fundamentales para resolver
problemas de procesamiento de grafos, donde el BFS ``es el algoritmo
clásico para encontrar el camino más corto entre dos nodos específicos''
(p.~13), mientras que el DFS explora conexiones profundas mediante
recursividad. El algoritmo BFS se utiliza para identificar el camino más
corto entre dos usuarios, como los amigos en común más cercanos,
mientras que el algoritmo DFS se emplea para explorar comunidades o
conexiones más profundas, como los seguidores de seguidores. La
implementación basada en programación orientada a objetos (POO) aporta
escalabilidad al sistema, al permitir la incorporación de atributos a
los nodos, como intereses o ubicación, así como la asignación de pesos a
las aristas, como la frecuencia de interacción.(Sánchez Torrubia \&
Gutiérrez Revenga, 2006)

\section{DESAROLLO}\label{desarollo}

\subsection{Importación de
librerías}\label{importaciuxf3n-de-libreruxedas}

Vamos a importar las librerías necesarias para:

\begin{itemize}
\item
  \texttt{collections.deque}: Para implementar colas eficientes en BFS
\item
  \texttt{networkx} y \texttt{matplotlib}: Para visualización de grafos
\item
  \texttt{ipywidgets}: Para interactividad en el notebook
\end{itemize}

\begin{Shaded}
\begin{Highlighting}[]
\ImportTok{from}\NormalTok{ collections }\ImportTok{import}\NormalTok{ deque}
\ImportTok{import}\NormalTok{ networkx }\ImportTok{as}\NormalTok{ nx}
\ImportTok{import}\NormalTok{ matplotlib.pyplot }\ImportTok{as}\NormalTok{ plt}
\ImportTok{from}\NormalTok{ IPython.display }\ImportTok{import}\NormalTok{ display}
\ImportTok{import}\NormalTok{ ipywidgets }\ImportTok{as}\NormalTok{ widgets}

\CommentTok{\# Configurar matplotlib para mostrar ventanas emergentes}
\OperatorTok{\%}\NormalTok{matplotlib qt}
\end{Highlighting}
\end{Shaded}

\subsection{Clases Base: Arista y Nodo}\label{clases-base-arista-y-nodo}

Definimos las estructuras fundamentales:

\begin{itemize}
\item
  Arista: Conexión entre nodos con peso.
\item
  Nodo: Elemento básico del grafo con nombre y tipo.
\end{itemize}

\begin{Shaded}
\begin{Highlighting}[]
\KeywordTok{class}\NormalTok{ Arista:}
    \KeywordTok{def} \FunctionTok{\_\_init\_\_}\NormalTok{(}\VariableTok{self}\NormalTok{, nodo1, nodo2, peso}\OperatorTok{=}\DecValTok{1}\NormalTok{):}
        \VariableTok{self}\NormalTok{.nodo1 }\OperatorTok{=}\NormalTok{ nodo1}
        \VariableTok{self}\NormalTok{.nodo2 }\OperatorTok{=}\NormalTok{ nodo2}
        \VariableTok{self}\NormalTok{.peso }\OperatorTok{=}\NormalTok{ peso}

\KeywordTok{class}\NormalTok{ Nodo:}
    \KeywordTok{def} \FunctionTok{\_\_init\_\_}\NormalTok{(}\VariableTok{self}\NormalTok{, nombre, tipo}\OperatorTok{=}\VariableTok{None}\NormalTok{):}
        \VariableTok{self}\NormalTok{.nombre }\OperatorTok{=}\NormalTok{ nombre}
        \VariableTok{self}\NormalTok{.tipo }\OperatorTok{=}\NormalTok{ tipo  }\CommentTok{\# entrada, salida, edificio, etc.}

    \KeywordTok{def} \FunctionTok{\_\_repr\_\_}\NormalTok{(}\VariableTok{self}\NormalTok{):}
        \ControlFlowTok{return} \VariableTok{self}\NormalTok{.nombre}
\end{Highlighting}
\end{Shaded}

\subsection{Clase Grafo: Estructura
Principal}\label{clase-grafo-estructura-principal}

Implementación con:

\begin{itemize}
\item
  Lista de adyacencia para almacenar conexiones
\item
  Métodos para agregar nodos y aristas
\item
  Algoritmos BFS y DFS para búsqueda de rutas
\end{itemize}

\begin{Shaded}
\begin{Highlighting}[]
\KeywordTok{class}\NormalTok{ Grafo:}
    \KeywordTok{def} \FunctionTok{\_\_init\_\_}\NormalTok{(}\VariableTok{self}\NormalTok{):}
        \VariableTok{self}\NormalTok{.nodos }\OperatorTok{=}\NormalTok{ \{\}}
        \VariableTok{self}\NormalTok{.aristas }\OperatorTok{=}\NormalTok{ []}
        \VariableTok{self}\NormalTok{.adj\_list }\OperatorTok{=}\NormalTok{ \{\}}

    \KeywordTok{def}\NormalTok{ agregar\_nodo(}\VariableTok{self}\NormalTok{, nodo):}
        \ControlFlowTok{if}\NormalTok{ nodo.nombre }\KeywordTok{not} \KeywordTok{in} \VariableTok{self}\NormalTok{.nodos:}
            \VariableTok{self}\NormalTok{.nodos[nodo.nombre] }\OperatorTok{=}\NormalTok{ nodo}
            \VariableTok{self}\NormalTok{.adj\_list[nodo] }\OperatorTok{=}\NormalTok{ []}

    \KeywordTok{def}\NormalTok{ agregar\_arista(}\VariableTok{self}\NormalTok{, nodo1, nodo2, peso}\OperatorTok{=}\DecValTok{1}\NormalTok{):}
        \ControlFlowTok{if}\NormalTok{ nodo1.nombre }\KeywordTok{in} \VariableTok{self}\NormalTok{.nodos }\KeywordTok{and}\NormalTok{ nodo2.nombre }\KeywordTok{in} \VariableTok{self}\NormalTok{.nodos:}
\NormalTok{            arista }\OperatorTok{=}\NormalTok{ Arista(nodo1, nodo2, peso)}
            \VariableTok{self}\NormalTok{.aristas.append(arista)}
            \VariableTok{self}\NormalTok{.adj\_list[nodo1].append(nodo2)}
            \VariableTok{self}\NormalTok{.adj\_list[nodo2].append(nodo1)}
        \ControlFlowTok{else}\NormalTok{:}
            \ControlFlowTok{raise} \PreprocessorTok{ValueError}\NormalTok{(}\StringTok{"Uno o ambos nodos no existen"}\NormalTok{)}
\end{Highlighting}
\end{Shaded}

\subsection{Algoritmo BFS (Breadth-First
Search)}\label{algoritmo-bfs-breadth-first-search}

\begin{itemize}
\item
  Encuentra la ruta más corta
\item
  Usa cola (FIFO) para explorar niveles
\end{itemize}

\begin{Shaded}
\begin{Highlighting}[]

\KeywordTok{def}\NormalTok{ bfs(}\VariableTok{self}\NormalTok{, inicio, fin):}
\NormalTok{        visitados }\OperatorTok{=}\NormalTok{ \{nodo: }\VariableTok{False} \ControlFlowTok{for}\NormalTok{ nodo }\KeywordTok{in} \VariableTok{self}\NormalTok{.nodos.values()\}}
\NormalTok{        cola }\OperatorTok{=}\NormalTok{ deque([(inicio, [inicio])])}
\NormalTok{        visitados[inicio] }\OperatorTok{=} \VariableTok{True}

        \ControlFlowTok{while}\NormalTok{ cola:}
\NormalTok{            actual, camino }\OperatorTok{=}\NormalTok{ cola.popleft()}
            \ControlFlowTok{if}\NormalTok{ actual }\OperatorTok{==}\NormalTok{ fin:}
                \ControlFlowTok{return}\NormalTok{ camino}
            \ControlFlowTok{for}\NormalTok{ vecino }\KeywordTok{in} \VariableTok{self}\NormalTok{.adj\_list[actual]:}
                \ControlFlowTok{if} \KeywordTok{not}\NormalTok{ visitados[vecino]:}
\NormalTok{                    visitados[vecino] }\OperatorTok{=} \VariableTok{True}
\NormalTok{                    cola.append((vecino, camino }\OperatorTok{+}\NormalTok{ [vecino]))}
        \ControlFlowTok{return} \VariableTok{None}
\end{Highlighting}
\end{Shaded}

\subsection{Algoritmo DFS (Depth-First
Search)}\label{algoritmo-dfs-depth-first-search}

\begin{itemize}
\item
  Explora ramas completas antes de retroceder
\item
  Implementación recursiva
\item
  Útil para encontrar componentes conectados
\end{itemize}

\begin{Shaded}
\begin{Highlighting}[]

\KeywordTok{def}\NormalTok{ dfs(}\VariableTok{self}\NormalTok{, inicio, destino):}
        \BuiltInTok{print}\NormalTok{(}\StringTok{"}\CharTok{\textbackslash{}n}\StringTok{Iniciando DFS (recursivo)..."}\NormalTok{)}
\NormalTok{        searched }\OperatorTok{=}\NormalTok{ \{nodo: }\VariableTok{False} \ControlFlowTok{for}\NormalTok{ nodo }\KeywordTok{in} \VariableTok{self}\NormalTok{.nodos.values()\}}
\NormalTok{        parents }\OperatorTok{=}\NormalTok{ \{\}}
\NormalTok{        componentR }\OperatorTok{=}\NormalTok{ []}

        \KeywordTok{def}\NormalTok{ \_dfs(nodo, parent}\OperatorTok{=}\VariableTok{None}\NormalTok{):}
\NormalTok{            componentR.append(nodo)}
\NormalTok{            searched[nodo] }\OperatorTok{=} \VariableTok{True}
\NormalTok{            parents[nodo] }\OperatorTok{=}\NormalTok{ parent}

            \BuiltInTok{print}\NormalTok{(}\SpecialStringTok{f"Estado de búsqueda en nodo }\SpecialCharTok{\{}\NormalTok{nodo}\SpecialCharTok{\}}\SpecialStringTok{:"}\NormalTok{)}
            \BuiltInTok{print}\NormalTok{(\{n.nombre: v }\ControlFlowTok{for}\NormalTok{ n, v }\KeywordTok{in}\NormalTok{ searched.items()\})}
            \BuiltInTok{print}\NormalTok{(}\SpecialStringTok{f"Vecinos de }\SpecialCharTok{\{}\NormalTok{nodo}\SpecialCharTok{\}}\SpecialStringTok{: }\SpecialCharTok{\{}\VariableTok{self}\SpecialCharTok{.}\NormalTok{adj\_list[nodo]}\SpecialCharTok{\}}\SpecialStringTok{"}\NormalTok{)}
            \BuiltInTok{print}\NormalTok{()}

            \ControlFlowTok{if}\NormalTok{ nodo }\OperatorTok{==}\NormalTok{ destino:}
                \ControlFlowTok{return} \VariableTok{True}

            \ControlFlowTok{for}\NormalTok{ vecino }\KeywordTok{in} \VariableTok{self}\NormalTok{.adj\_list[nodo]:}
                \ControlFlowTok{if} \KeywordTok{not}\NormalTok{ searched[vecino]:}
                    \ControlFlowTok{if}\NormalTok{ \_dfs(vecino, nodo):}
                        \ControlFlowTok{return} \VariableTok{True}
                    \BuiltInTok{print}\NormalTok{(}\SpecialStringTok{f"Finaliza }\SpecialCharTok{\{}\NormalTok{vecino}\SpecialCharTok{\}}\SpecialStringTok{"}\NormalTok{)}
                    \BuiltInTok{print}\NormalTok{(}\SpecialStringTok{f"Vuelve a }\SpecialCharTok{\{}\NormalTok{nodo}\SpecialCharTok{\}}\SpecialStringTok{"}\NormalTok{)}
                    \BuiltInTok{print}\NormalTok{()}

            \ControlFlowTok{return} \VariableTok{False}

\NormalTok{        encontrado }\OperatorTok{=}\NormalTok{ \_dfs(inicio)}

        \ControlFlowTok{if}\NormalTok{ encontrado:}
\NormalTok{            camino }\OperatorTok{=}\NormalTok{ []}
\NormalTok{            actual }\OperatorTok{=}\NormalTok{ destino}
            \ControlFlowTok{while}\NormalTok{ actual }\KeywordTok{is} \KeywordTok{not} \VariableTok{None}\NormalTok{:}
\NormalTok{                camino.append(actual)}
\NormalTok{                actual }\OperatorTok{=}\NormalTok{ parents.get(actual)}
\NormalTok{            camino.reverse()}
            \BuiltInTok{print}\NormalTok{(}\StringTok{" Ruta encontrada con DFS:"}\NormalTok{)}
            \BuiltInTok{print}\NormalTok{(}\StringTok{" → "}\NormalTok{.join(n.nombre }\ControlFlowTok{for}\NormalTok{ n }\KeywordTok{in}\NormalTok{ camino))}
            \ControlFlowTok{return}\NormalTok{ camino}
        \ControlFlowTok{else}\NormalTok{:}
            \BuiltInTok{print}\NormalTok{(}\StringTok{" No se encontró ruta con DFS"}\NormalTok{)}
            \ControlFlowTok{return} \VariableTok{None}
\end{Highlighting}
\end{Shaded}

\subsection{Función de Construcción del Grafo
Demo}\label{funciuxf3n-de-construcciuxf3n-del-grafo-demo}

Crea un grafo predefinido que simula una red social con:

\begin{itemize}
\item
  14 nodos (usuarios)
\item
  20 conexiones bidireccionales
\end{itemize}

\begin{Shaded}
\begin{Highlighting}[]
\KeywordTok{def}\NormalTok{ construir\_grafo\_demo():}
\NormalTok{    grafo }\OperatorTok{=}\NormalTok{ Grafo()}
\NormalTok{    nodos }\OperatorTok{=}\NormalTok{ \{}
        \StringTok{"Jairo"}\NormalTok{: Nodo(}\StringTok{"Jairo"}\NormalTok{, }\StringTok{"entrada"}\NormalTok{),}
        \StringTok{"Marco"}\NormalTok{: Nodo(}\StringTok{"Marco"}\NormalTok{, }\StringTok{"punto\_interes"}\NormalTok{),}
        \StringTok{"Daniela"}\NormalTok{: Nodo(}\StringTok{"Daniela"}\NormalTok{, }\StringTok{"salida"}\NormalTok{),}
        \StringTok{"Camila"}\NormalTok{: Nodo(}\StringTok{"Camila"}\NormalTok{, }\StringTok{"punto\_interes"}\NormalTok{),}
        \StringTok{"Eve"}\NormalTok{: Nodo(}\StringTok{"Eve"}\NormalTok{, }\StringTok{"punto\_interes"}\NormalTok{),}
        \StringTok{"Richard"}\NormalTok{: Nodo(}\StringTok{"Richard"}\NormalTok{, }\StringTok{"punto\_interes"}\NormalTok{),}
        \StringTok{"Grace"}\NormalTok{: Nodo(}\StringTok{"Grace"}\NormalTok{, }\StringTok{"edificio"}\NormalTok{),}
        \StringTok{"Estefano"}\NormalTok{: Nodo(}\StringTok{"Estefano"}\NormalTok{, }\StringTok{"salida"}\NormalTok{),}
        \StringTok{"Lenin"}\NormalTok{: Nodo(}\StringTok{"Lenin"}\NormalTok{, }\StringTok{"usuario"}\NormalTok{),}
        \StringTok{"Lorena"}\NormalTok{: Nodo(}\StringTok{"Lorena"}\NormalTok{, }\StringTok{"usuario"}\NormalTok{),}
        \StringTok{"Eddy"}\NormalTok{: Nodo(}\StringTok{"Eddy"}\NormalTok{, }\StringTok{"usuario"}\NormalTok{),}
        \StringTok{"Laura"}\NormalTok{: Nodo(}\StringTok{"Laura"}\NormalTok{, }\StringTok{"usuario"}\NormalTok{),}
        \StringTok{"Miguel"}\NormalTok{: Nodo(}\StringTok{"Miguel"}\NormalTok{, }\StringTok{"usuario"}\NormalTok{),}
        \StringTok{"Tamara"}\NormalTok{: Nodo(}\StringTok{"Tamara"}\NormalTok{, }\StringTok{"usuario"}\NormalTok{)}
\NormalTok{    \}}

    \ControlFlowTok{for}\NormalTok{ nodo }\KeywordTok{in}\NormalTok{ nodos.values():}
\NormalTok{        grafo.agregar\_nodo(nodo)}

\NormalTok{    conexiones }\OperatorTok{=}\NormalTok{ [}
\NormalTok{        (}\StringTok{"Jairo"}\NormalTok{, }\StringTok{"Eve"}\NormalTok{),}
\NormalTok{        (}\StringTok{"Eve"}\NormalTok{, }\StringTok{"Daniela"}\NormalTok{),}
\NormalTok{        (}\StringTok{"Camila"}\NormalTok{, }\StringTok{"Richard"}\NormalTok{),}
\NormalTok{        (}\StringTok{"Richard"}\NormalTok{, }\StringTok{"Grace"}\NormalTok{),}
\NormalTok{        (}\StringTok{"Grace"}\NormalTok{, }\StringTok{"Daniela"}\NormalTok{),}
\NormalTok{        (}\StringTok{"Jairo"}\NormalTok{, }\StringTok{"Marco"}\NormalTok{),}
\NormalTok{        (}\StringTok{"Marco"}\NormalTok{, }\StringTok{"Estefano"}\NormalTok{),}
\NormalTok{        (}\StringTok{"Estefano"}\NormalTok{, }\StringTok{"Richard"}\NormalTok{),}
\NormalTok{        (}\StringTok{"Estefano"}\NormalTok{, }\StringTok{"Eve"}\NormalTok{),}
\NormalTok{        (}\StringTok{"Estefano"}\NormalTok{, }\StringTok{"Camila"}\NormalTok{),}
\NormalTok{        (}\StringTok{"Lenin"}\NormalTok{, }\StringTok{"Marco"}\NormalTok{),}
\NormalTok{        (}\StringTok{"Lenin"}\NormalTok{, }\StringTok{"Lorena"}\NormalTok{),}
\NormalTok{        (}\StringTok{"Lorena"}\NormalTok{, }\StringTok{"Eddy"}\NormalTok{),}
\NormalTok{        (}\StringTok{"Eddy"}\NormalTok{, }\StringTok{"Laura"}\NormalTok{),}
\NormalTok{        (}\StringTok{"Laura"}\NormalTok{, }\StringTok{"Eve"}\NormalTok{),}
\NormalTok{        (}\StringTok{"Miguel"}\NormalTok{, }\StringTok{"Eddy"}\NormalTok{),}
\NormalTok{        (}\StringTok{"Miguel"}\NormalTok{, }\StringTok{"Richard"}\NormalTok{),}
\NormalTok{        (}\StringTok{"Tamara"}\NormalTok{, }\StringTok{"Lorena"}\NormalTok{),}
\NormalTok{        (}\StringTok{"Tamara"}\NormalTok{, }\StringTok{"Camila"}\NormalTok{),}
\NormalTok{        (}\StringTok{"Grace"}\NormalTok{, }\StringTok{"Tamara"}\NormalTok{)}
\NormalTok{    ]}

    \ControlFlowTok{for}\NormalTok{ a, b }\KeywordTok{in}\NormalTok{ conexiones:}
\NormalTok{        grafo.agregar\_arista(nodos[a], nodos[b])}

    \ControlFlowTok{return}\NormalTok{ grafo}
\end{Highlighting}
\end{Shaded}

\subsection{Visualización Interactiva del
Grafo}\label{visualizaciuxf3n-interactiva-del-grafo}

Función para dibujar el grafo usando networkx y matplotlib, con
capacidad para seleccionar nodos haciendo clic. Características:

\begin{itemize}
\item
  Layout Kamada-Kawai para distribución orgánica
\item
  Eventos de clic para selección interactiva
\item
  Temporización para mantener la figura activa
\end{itemize}

\begin{Shaded}
\begin{Highlighting}[]
\KeywordTok{def}\NormalTok{ dibujar\_grafo\_interactivo(grafo):}
\NormalTok{    G }\OperatorTok{=}\NormalTok{ nx.Graph()}
    \ControlFlowTok{for}\NormalTok{ nodo }\KeywordTok{in}\NormalTok{ grafo.nodos.values():}
\NormalTok{        G.add\_node(nodo.nombre)}
    \ControlFlowTok{for}\NormalTok{ arista }\KeywordTok{in}\NormalTok{ grafo.aristas:}
\NormalTok{        G.add\_edge(arista.nodo1.nombre, arista.nodo2.nombre)}

\NormalTok{    pos }\OperatorTok{=}\NormalTok{ nx.kamada\_kawai\_layout(G)}
\NormalTok{    fig, ax }\OperatorTok{=}\NormalTok{ plt.subplots(figsize}\OperatorTok{=}\NormalTok{(}\DecValTok{14}\NormalTok{, }\DecValTok{10}\NormalTok{))}
\NormalTok{    nx.draw(G, pos, with\_labels}\OperatorTok{=}\VariableTok{True}\NormalTok{, node\_color}\OperatorTok{=}\StringTok{\textquotesingle{}lightblue\textquotesingle{}}\NormalTok{, edge\_color}\OperatorTok{=}\StringTok{\textquotesingle{}gray\textquotesingle{}}\NormalTok{,}
\NormalTok{            node\_size}\OperatorTok{=}\DecValTok{2000}\NormalTok{, font\_size}\OperatorTok{=}\DecValTok{10}\NormalTok{, width}\OperatorTok{=}\DecValTok{2}\NormalTok{)}
    
\NormalTok{    plt.title(}\StringTok{" Red Social {-} Selecciona dos usuarios para buscar la ruta"}\NormalTok{)}

\NormalTok{    seleccionados }\OperatorTok{=}\NormalTok{ []}

    \KeywordTok{def}\NormalTok{ onclick(event):}
        \ControlFlowTok{if}\NormalTok{ event.inaxes }\KeywordTok{is} \KeywordTok{not}\NormalTok{ ax:}
            \ControlFlowTok{return}
\NormalTok{        x, y }\OperatorTok{=}\NormalTok{ event.xdata, event.ydata}
\NormalTok{        nodo\_cercano }\OperatorTok{=} \BuiltInTok{min}\NormalTok{(pos, key}\OperatorTok{=}\KeywordTok{lambda}\NormalTok{ n: (pos[n][}\DecValTok{0}\NormalTok{]}\OperatorTok{{-}}\NormalTok{x)}\OperatorTok{**}\DecValTok{2} \OperatorTok{+}\NormalTok{ (pos[n][}\DecValTok{1}\NormalTok{]}\OperatorTok{{-}}\NormalTok{y)}\OperatorTok{**}\DecValTok{2}\NormalTok{)}
        \ControlFlowTok{if}\NormalTok{ nodo\_cercano }\KeywordTok{not} \KeywordTok{in}\NormalTok{ seleccionados:}
\NormalTok{            seleccionados.append(nodo\_cercano)}
            \BuiltInTok{print}\NormalTok{(}\SpecialStringTok{f"Seleccionado: }\SpecialCharTok{\{}\NormalTok{nodo\_cercano}\SpecialCharTok{\}}\SpecialStringTok{"}\NormalTok{)}
        \ControlFlowTok{if} \BuiltInTok{len}\NormalTok{(seleccionados) }\OperatorTok{==} \DecValTok{2}\NormalTok{:}
\NormalTok{            fig.canvas.mpl\_disconnect(cid)}
\NormalTok{            plt.close(fig)}

\NormalTok{    cid }\OperatorTok{=}\NormalTok{ fig.canvas.mpl\_connect(}\StringTok{\textquotesingle{}button\_press\_event\textquotesingle{}}\NormalTok{, onclick)}
\NormalTok{    plt.show()}

    \ControlFlowTok{while}\NormalTok{ plt.fignum\_exists(fig.number):}
\NormalTok{        plt.pause(}\FloatTok{0.1}\NormalTok{)}

    \ControlFlowTok{return}\NormalTok{ seleccionados}
\end{Highlighting}
\end{Shaded}

\subsection{Visualización de Rutas}\label{visualizaciuxf3n-de-rutas}

Función para resaltar rutas encontradas:

\begin{itemize}
\item
  Nodos en rojo: Ruta encontrada
\item
  Aristas en rojo: Conexiones utilizadas
\item
  Diferenciación visual entre BFS/DFS
\end{itemize}

\begin{Shaded}
\begin{Highlighting}[]
\KeywordTok{def}\NormalTok{ dibujar\_ruta(grafo, ruta, titulo):}
\NormalTok{    G }\OperatorTok{=}\NormalTok{ nx.Graph()}
    \ControlFlowTok{for}\NormalTok{ nodo }\KeywordTok{in}\NormalTok{ grafo.nodos.values():}
\NormalTok{        G.add\_node(nodo.nombre)}
    \ControlFlowTok{for}\NormalTok{ arista }\KeywordTok{in}\NormalTok{ grafo.aristas:}
\NormalTok{        G.add\_edge(arista.nodo1.nombre, arista.nodo2.nombre)}

\NormalTok{    pos }\OperatorTok{=}\NormalTok{ nx.kamada\_kawai\_layout(G)}

\NormalTok{    fig, ax }\OperatorTok{=}\NormalTok{ plt.subplots(figsize}\OperatorTok{=}\NormalTok{(}\DecValTok{14}\NormalTok{, }\DecValTok{10}\NormalTok{))}

\NormalTok{    nombres\_ruta }\OperatorTok{=}\NormalTok{ [n.nombre }\ControlFlowTok{for}\NormalTok{ n }\KeywordTok{in}\NormalTok{ ruta]}
\NormalTok{    colores\_nodos }\OperatorTok{=}\NormalTok{ [}\StringTok{\textquotesingle{}red\textquotesingle{}} \ControlFlowTok{if}\NormalTok{ nodo }\KeywordTok{in}\NormalTok{ nombres\_ruta }\ControlFlowTok{else} \StringTok{\textquotesingle{}lightblue\textquotesingle{}} \ControlFlowTok{for}\NormalTok{ nodo }\KeywordTok{in}\NormalTok{ G.nodes()]}
\NormalTok{    aristas\_ruta }\OperatorTok{=} \BuiltInTok{list}\NormalTok{(}\BuiltInTok{zip}\NormalTok{(nombres\_ruta[:}\OperatorTok{{-}}\DecValTok{1}\NormalTok{], nombres\_ruta[}\DecValTok{1}\NormalTok{:]))}
\NormalTok{    colores\_aristas }\OperatorTok{=}\NormalTok{ [}\StringTok{\textquotesingle{}red\textquotesingle{}} \ControlFlowTok{if}\NormalTok{ (u, v) }\KeywordTok{in}\NormalTok{ aristas\_ruta }\KeywordTok{or}\NormalTok{ (v, u) }\KeywordTok{in}\NormalTok{ aristas\_ruta }\ControlFlowTok{else} \StringTok{\textquotesingle{}gray\textquotesingle{}} \ControlFlowTok{for}\NormalTok{ u, v }\KeywordTok{in}\NormalTok{ G.edges()]}

\NormalTok{    nx.draw(G, pos, with\_labels}\OperatorTok{=}\VariableTok{True}\NormalTok{, node\_color}\OperatorTok{=}\NormalTok{colores\_nodos, edge\_color}\OperatorTok{=}\NormalTok{colores\_aristas,}
\NormalTok{            node\_size}\OperatorTok{=}\DecValTok{2000}\NormalTok{, font\_size}\OperatorTok{=}\DecValTok{10}\NormalTok{, width}\OperatorTok{=}\DecValTok{2}\NormalTok{, ax}\OperatorTok{=}\NormalTok{ax)}

    \CommentTok{\# Título informativo con enfoque de red social y tipo de algoritmo}
    \ControlFlowTok{if} \StringTok{"BFS"} \KeywordTok{in}\NormalTok{ titulo:}
\NormalTok{        plt.title(}\StringTok{" BFS {-} Ruta más corta entre dos usuarios en la red social"}\NormalTok{)}
    \ControlFlowTok{elif} \StringTok{"DFS"} \KeywordTok{in}\NormalTok{ titulo:}
\NormalTok{        plt.title(}\StringTok{"🔶 DFS {-} Ruta profunda entre dos usuarios en la red social"}\NormalTok{)}
    \ControlFlowTok{else}\NormalTok{:}
\NormalTok{        plt.title(}\SpecialStringTok{f" Ruta encontrada {-} }\SpecialCharTok{\{}\NormalTok{titulo}\SpecialCharTok{\}}\SpecialStringTok{"}\NormalTok{)}

\NormalTok{    plt.show()}
\end{Highlighting}
\end{Shaded}

\subsection{Función Principal de
Simulación}\label{funciuxf3n-principal-de-simulaciuxf3n}

Orquesta todo el flujo:

\begin{itemize}
\item
  Construye el grafo demo
\item
  Permite selección interactiva
\item
  Ejecuta BFS y DFS
\item
  Visualiza resultados
\end{itemize}

\begin{Shaded}
\begin{Highlighting}[]
\KeywordTok{def}\NormalTok{ ejecutar\_simulacion():}
\NormalTok{    grafo }\OperatorTok{=}\NormalTok{ construir\_grafo\_demo()}
    
    \BuiltInTok{print}\NormalTok{(}\StringTok{" Haz clic en el nodo de inicio y luego en el de destino..."}\NormalTok{)}
\NormalTok{    seleccionados }\OperatorTok{=}\NormalTok{ dibujar\_grafo\_interactivo(grafo)}
    
    \ControlFlowTok{if} \BuiltInTok{len}\NormalTok{(seleccionados) }\OperatorTok{!=} \DecValTok{2}\NormalTok{:}
        \BuiltInTok{print}\NormalTok{(}\StringTok{"❌ Debes seleccionar exactamente 2 nodos"}\NormalTok{)}
        \ControlFlowTok{return}
    
\NormalTok{    inicio, fin }\OperatorTok{=}\NormalTok{ seleccionados}
\NormalTok{    nodo\_inicio }\OperatorTok{=}\NormalTok{ grafo.nodos[inicio]}
\NormalTok{    nodo\_fin }\OperatorTok{=}\NormalTok{ grafo.nodos[fin]}
    
    \BuiltInTok{print}\NormalTok{(}\SpecialStringTok{f"}\CharTok{\textbackslash{}n}\SpecialStringTok{Ruta desde }\SpecialCharTok{\{}\NormalTok{inicio}\SpecialCharTok{\}}\SpecialStringTok{ hasta }\SpecialCharTok{\{}\NormalTok{fin}\SpecialCharTok{\}}\SpecialStringTok{:"}\NormalTok{)}
    
    \BuiltInTok{print}\NormalTok{(}\StringTok{"}\CharTok{\textbackslash{}n}\StringTok{ Ejecutando BFS (Ruta más corta)..."}\NormalTok{)}
\NormalTok{    ruta\_bfs }\OperatorTok{=}\NormalTok{ grafo.bfs(nodo\_inicio, nodo\_fin)}
    \ControlFlowTok{if}\NormalTok{ ruta\_bfs:}
        \BuiltInTok{print}\NormalTok{(}\StringTok{" → "}\NormalTok{.join(n.nombre }\ControlFlowTok{for}\NormalTok{ n }\KeywordTok{in}\NormalTok{ ruta\_bfs))}
\NormalTok{        dibujar\_ruta(grafo, ruta\_bfs, }\StringTok{"BFS {-} Ruta más corta"}\NormalTok{)}
    \ControlFlowTok{else}\NormalTok{:}
        \BuiltInTok{print}\NormalTok{(}\StringTok{"No se encontró ruta con BFS"}\NormalTok{)}
    
    \BuiltInTok{print}\NormalTok{(}\StringTok{"}\CharTok{\textbackslash{}n}\StringTok{🔶 Ejecutando DFS (Ruta más profunda)..."}\NormalTok{)}
\NormalTok{    ruta\_dfs }\OperatorTok{=}\NormalTok{ grafo.dfs(nodo\_inicio, nodo\_fin)}
    \ControlFlowTok{if}\NormalTok{ ruta\_dfs:}
        \BuiltInTok{print}\NormalTok{(}\StringTok{" → "}\NormalTok{.join(n.nombre }\ControlFlowTok{for}\NormalTok{ n }\KeywordTok{in}\NormalTok{ ruta\_dfs))}
\NormalTok{        dibujar\_ruta(grafo, ruta\_dfs, }\StringTok{"DFS {-} Ruta más profunda"}\NormalTok{)}
    \ControlFlowTok{else}\NormalTok{:}
        \BuiltInTok{print}\NormalTok{(}\StringTok{"No se encontró ruta con DFS"}\NormalTok{)}

\CommentTok{\# Ejecutar la simulación}
\BuiltInTok{print}\NormalTok{(}\StringTok{"=== Simulador de Conexiones en Red Social ==="}\NormalTok{)}
\NormalTok{ejecutar\_simulacion()}
\end{Highlighting}
\end{Shaded}

\subsection{Resultados}\label{resultados}

\begin{itemize}
\tightlist
\item
  BFS
\end{itemize}

\begin{figure}[H]

{\centering \pandocbounded{\includegraphics[keepaspectratio]{attachment:image.png}}

}

\caption{Ruta corta entre Daniela y Eddy}

\end{figure}%

\phantomsection\label{refs}
\begin{CSLReferences}{1}{0}
\bibitem[\citeproctext]{ref-netto2003grafos}
Netto, P. O. B. (2003). \emph{Grafos: teoria, modelos, algoritmos}.
Editora Blucher.

\bibitem[\citeproctext]{ref-sanchez2006tutorial}
Sánchez Torrubia, M. G., \& Gutiérrez Revenga, S. (2006). \emph{Tutorial
interactivo para la ense{ñ}anza y el aprendizaje de los algoritmos de
b{ú}squeda en anchura y en profundidad}.

\end{CSLReferences}




\end{document}
